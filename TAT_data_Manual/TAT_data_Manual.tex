\documentclass[12pt]{article}    %text size is 12pt and article form 
\usepackage{enumitem}            %To use item and enumerate
\usepackage{graphicx}            %To resize the table 
\usepackage{geometry}            % To set the paper size
\usepackage{hyperref}
\geometry{a4paper,margin=1in}    %set paper A4 size and margin is 1 
\linespread{1.2}         %line space is 1.2
\title{TAT Database Program Manual}  % give the information for title
\date{July 2018}

%If I use \verb, it means the origin form in terminal 
%If I use {\it}, it means the name of something ex: file or table

\begin{document}
 \maketitle     % create title
 \newpage		 
	
 \tableofcontents	  % create the contents
 \newpage    
		
 % The first section. Describe the database TAT	
 \section{Description of Database {\it TAT}}

 \begin{itemize}
  \item Database {\it TAT} records data for Taiwan Automated Telescope.
  \item Database {\it TAT} contains three tables:\\
   {\it targets, data\_file, observatory}
 \end{itemize}
 	
 % describe the Table targets
 \subsection{Table {\it targets}}
 \begin{itemize}
  \item Table {\it targets} shows information of the targets for observation.
  \item Table {\it targets} contains the following keys:\\
  \footnotesize   % lower size for keys
  \indent ID, NAME, RA(deg), DEC(deg), RA, DEC, MAGNITUDE, PERIOD, TYPE, BFE0, F0, BFE1, F1, BFE2, F2, BFE3, F3, BFE4, F4, BFE5, F5, BFE6, F6
  \normalsize
  \item The meaning of each key is:
  \begin{itemize}
   \item \textbf {ID} is the identification number for every data and it is unique.
   \item \textbf {NAME} is the name of target and it is unique.
   \item \textbf {RA(deg)} is the right ascension of the target and its unit is degree.
   \item \textbf {DEC(deg)} is the declination of the target and its unit is degree.
   \item \textbf {RA} is the right ascension of the target and its type is HH:MM:SS.
   \item \textbf {DEC} is the declination of the target and its type is DEG:ARCMIN:ARCSEC. 
   \item \textbf {MAGNITUDE} is the cataloged relative magnitude of the target.
   \item \textbf {PERIOD} is the period of magnitude oscillation. 
   \item \textbf {TYPE} is the type of target. Example: star, galaxy...
   \item \textbf {BFE0,1,2,3,...} is the best exposure time for filter 0,1,2,3,...
   \item \textbf {F0,1,2,3,...} is the filter 0,1,2,3,...
  \end{itemize}
  
	
  \item The following table is an example:
  % create the table	
  \begin{table}[!htbp]
   \centering
   \caption{Example for table {\it targets}}
   \resizebox{\textwidth}{!}{
   \begin{tabular}{|*{21}{c|}}
    \hline
    ID & NAME & RA(deg) & DEC(deg) & RA & DEC & MAGNITUDE & PERIOD & TYPE & BFE0 & F0 & BFE1 & F1 & BFE2 & F2 & BFE3 & F3 & BFE4 & F4 & BFE5 & F5   \\ \hline
    1 & IC5146 & 328.35 & 47.266 & 21:53:24 & 47:16:00 & 0 & 0 & star & 0 & A & 0 & B & 0 & C & 0 & N & 0 & R & 0 & V\\ \hline
   \end{tabular}
   }
  \end{table}
 \end{itemize}

 % describe the Table file_data
 \subsection{Table {\it data\_file}}
 \begin{itemize}
  \item Table {\it data\_file} shows the information of images.
  \item Table {\it data\_file} contains the following keys:\\
  \footnotesize
  \indent ID, FILENAME, FILEPATH, FILTER, RA(deg), DEC(deg), RA, DEC, SITENAME, CCDTEMP, EXPTIME, DATE-OBS, TIME-OBS, MJD-OBS, AIRMASS, JD, SUBBED, DIVFITTED
  \normalsize
  \item The meaning of each key is:
  \begin{itemize}
   \item {\bf ID} is the identification number for every data and it is unique.
   \item {\bf FILENAME} is the filename of image and it is unique.
   \item {\bf FILEPATH} is the path of data file.
   \item {\bf FILTER} is the one of filters in the band A, B, C, N, R, or V.
   \item {\bf RA(deg)} is the right ascension of the center of image and its unit is degree.
   \item {\bf DEC(deg)} is the declination of the center of image and its unit is degree.
   \item \textbf {RA} is the right ascension of the center of image and its type is HH:MM:SS.
   \item \textbf {DEC} is the declination of the center of image and its type is DEG:ARCMIN:ARCSEC.
   \item {\bf SITENAME} is the location of site.
   \item {\bf CCDTEMP} is the CCD temperature.
   \item {\bf EXPTIME} is the exposure time.
   \item {\bf DATE-OBS} is the date and its type is YYYY/MM/DD.
   \item {\bf TIME-OBS} is the observation time and its type is HH:MM:SS.SS
   \item {\bf MJD-OBS} the Modified Julian Date.
   \item {\bf AIRMASS} provides the condition to compare.
   \item {\bf JD} is the Julian Date.
   \item {\bf SUBBED} if the file has been subbed or not.
   \item {\bf FLATDIVED} if the file has been divided by flat or not.
   
  \end{itemize}
	
  \item The following table is an example:
 \end{itemize}
	
 %create the table
 \begin{table}[!htbp]
  \centering
  \caption{Example for table {\it data\_file}}
  \resizebox{\textwidth}{!}
  {
   \begin{tabular}{|*{18}{c|}}
    \hline
    ID & FILENAME & FILEPATH & FILTER & RA(deg) & DEC(deg) & RA & DEC & SITENAME & CCDTEMP & EXPTIME & DATE-OBS & TIME-OBS & MJD-OBS & AIRMASS & JD & subbed & divfitted  \\ \hline
    1 & AStarTF20180705\_215223.fit & /home2/TAT/data/raw/TF/image/20180705 & A & 0 & 0 & 19:20:30 & 11:02:01 & TF & -16.2883 & 600 & 2018-07-05 &	21:52:23.26 & 58304.918345 & NULL & 2458305.41834 & 0 & 0 \\ \hline 
    2 & AStarTF20180705\_221349.fit & /home2/TAT/data/raw/TF/image/20180705 & A & 0 & 0 & 19:20:30 & 11:02:01 & TF & -30.0856 & 600 & 2018-07-05 &	22:13:49.26 & 58304.933229 & NULL & 2458305.43323 & 0 & 0 \\ \hline
    3 & AStarTF20180705\_223518.fit & /home2/TAT/data/raw/TF/image/20180705 & A & 0 & 0 & 19:20:30 & 11:02:01 & TF & -30.0385 & 600 & 2018-07-05 & 	22:35:18.26 & 58304.94816 & NULL & 2458305.44816 & 0 & 0 \\ \hline
    4 & AStarTF20180705\_225646.fit & /home2/TAT/data/raw/TF/image/20180705 & A & 0 & 0 & 19:20:30 & 11:02:01 & TF & -30.0605 & 600 & 2018-07-05 &	22:56:46.26 & 58304.963056 & NULL & 2458305.46306 & 0 & 0 \\ \hline 
   \end{tabular}
   }
  \end{table}

 %describle the table observatory
 \subsection{Table {\it observatory}}
 \begin{itemize}
  \item Table {\it observatory} contains the following key:\\
  \footnotesize
  \indent ID, SITENAME, SITELAT, SITELONG, SITEALT
  \normalsize
  \item The meaning of each key:
  \begin{itemize}
   \item \textbf {ID} is the identification number for data and it is unique.
   \item \textbf {SITENAME} is the location of observatory and it is unique.
   \item \textbf {SITELAT} is the latitude of the observatory. 
   \item \textbf {SITELONG} is the longitude of the observatory.
   \item \textbf {SITEALT} is the altitude of the observatory.
  \end{itemize}
  
	
  \item The following table is an example:
 \end{itemize}
	
 %create the table	
 \begin{table}[!htbp]
  \centering
  \caption{Table {\it observatory}}
  \begin{tabular}{|*{5}{c|}}
   \hline
   ID & SITENAME & SITELAT & SITELONG & SITEALT \\ \hline
   1 & TF & 28.30 & -16.51 & 2300 \\ \hline    
   2 & LI-JIANG & 26.69 & 100.03 & 3330 \\ \hline 
  \end{tabular}
 \end{table}

 \newpage
	
 %The second section. 
 \section{Program {\it TAT\_database\_update}}
 \begin{itemize}
  \item This program updates the data of images which saved in the path written in the file {\it back\_up\_path.txt} in database {\it TAT}
 \end{itemize}
 
 \subsection{Pre-requirements}
 \begin{itemize}
  \item This program uses python 2.7.11 and mysql in CentOS 7.
  \begin{enumerate}
   \item Install python with:\\
   \verb|	yum install python2|
   \item Install python-pip with:\\
   \verb|		yum install epel-release|\\
   \verb|		yum install python-pip| 
   \item Install MariaDB
  
   \begin{enumerate}[itemindent=1pt,label=(\arabic*)]
    \item Install MariaDB with yum:\\
    \verb|		yum install mariadb-server mariadb|
    \item After the installation are completed, start MariaDB with:\\
    \verb|		systemctl start mariadb|    
   \end{enumerate}
  \end{enumerate}
 \end{itemize}
 \subsection{Install this program {\it TAT\_database\_update}}
 \begin{itemize}
  \item Download this program and follow the steps to install.
  \begin{enumerate}
   \item Download from github, using this command:\\
   \verb|		git clone https://github.com/yun-yan/TAT_db|
   \item Create the database {\it TAT}, using the command:\\
   \verb|		mysql < TAT_create_db.sql|	
   \item Install the modules for the file {\it update\_to\_TAT\_db.py}, the command:\\
   \verb|		pip install -r requirements.txt|
   \item Let the file {\it update\_to\_TAT\_db.py} be used in anywhere, the command:\\
   \verb|		make install|
 \end{enumerate}
 \end{itemize}	

 \subsection{Running}
 \begin{itemize}
  \item Insert the data of images stored in the path written in the file {\it back\_up\_path.txt} to database {\it TAT} with this command:\\
  \verb|		update_to_TAT_db.py|
 \end{itemize}
 	
 \subsection{Source Files}	
 \begin{itemize}
  \item The files are located in the path \verb|/home2/TAT/program/TAT_database|
  \item It contains the following files:
  \begin{itemize}
  \item \textbf {INSTALL} is the simple manual which describes how to install and execute.
  \item \textbf {README.md} illustrates what this program can do.
  \item \textbf {back\_up\_path.txt} contains the path you want to process.
  \item \textbf {update\_to\_TAT\_db.py} inserts the data of images in the path written into file {\it back\_up\_path.txt} to database {\it TAT}.
  \item \textbf {Makefile} is the convenient file to provide to use the command \verb|make| 
  \item \textbf {TAT\_create\_db.sql} is the file to create the database {\it TAT}.
  \item \textbf {requirement.txt} provides the modules for installation.
  \item \textbf {log.txt} records the path already checked.
  \end{itemize}
  
 \end{itemize}


 \subsection{Authority}
 \begin{itemize}
  \item {\it TAT@localhost} has the privileges of selecting and modifying the database {\it TAT}, and its password is 1234
  \item {\it read@localhost} has the only privilege of selecting database {\it TAT}, and its password is 1234
 \end{itemize}
 \subsection{Uninstall}
 \begin{itemize}
  \item Remove the file {\it update\_to\_TAT\_db.py} from \verb|/usr/local/bin|  with this command:\\
  \verb|		make uninstall|
  \item Clean the log from {\it log.txt} with:\\
  \verb|		make clean|
  \item Remove the database {\it TAT}, following the steps:
  \begin{enumerate}
   \item Enter mysql with this command:\\
   \verb|	mysql|
   \item Delete the database {\it TAT} with this command:\\
   \verb|	drop database TAT;| 
  \end{enumerate}			  
 \end{itemize}


\end{document}
